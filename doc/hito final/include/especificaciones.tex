\section{Simulación de Datos}

Para el desarrollo del algoritmo de identificación de líneas espectrales se utilizará el servicio web de datos simulados que proporcionará ChiVO. Este proyecto, llamado Astronomical SYntetic Data Observatory (ASYDO) se desarrolla en paralelo a este proyecto como parte de las herramientas que podrán utilizar los astrónomos como servicio web. Dicho proyecto está disponible en \footnote\url{https://github.com/ChileanVirtualObservatory/ASYDO}.


La simulación permitirá generar un set de entrenamiento para desarrollar el algoritmo de identificación propuesto en este proyecto. Al ser este servicio de simulación un input importante para el desarrollo del algoritmo, se han realizado una serie de reuniones para trabajar conjuntamente y obtener un set de datos simulados adecuado.

Con la ayuda de los astrónomos que participan en el proyecto ChiVO se ha determinado los parámetros necesarios para llevar a cabo la simulación. Se ha definido la complejidad y la importancia de replicar características determinadas que ayudarán a obtener curvas de espectros que se acerquen lo suficiente a los datos que se espera obtener a partir de las observaciones de ALMA.

Para que el algoritmo sea correctamente desarrollado con datos simulados, se deberá incluir en la meta-data de los cubos las líneas de emisión presentes en los espectros. Con esto será posible evaluar el modelo predictivo y determinar métricas para validar las predicciones.

Las características de las mediciones de ALMA han permitido obtener las siguientes especificaciones a la hora de simular cubos de datos tipo ALMA:

\begin {table}[H]
\begin{center}
	\begin{tabular}{|c|c|}
		\hline Ancho de banda espectral & 4000 MHz \\ 
		\hline Resolusión espectral & 1 MHz \\ 
		\hline 
	\end{tabular}
	\caption {Especificaciones comúnes para todos los cubos de datos simulados}
\end{center}
\end{table}

Y como parámetros de entrada, el usuario tendrá que proporcionar:

\begin {table}[H]
\begin{center}
	\begin{tabular}{|c|c|}
		\hline Frecuencia central &  MHz \\ 
		\hline Resolusión espectral & 1 MHz \\ 
		\hline Asención &  Grados \\ 
		\hline Declinación & Grados \\ 
		\hline Ancho de las lineas & (fwhm) \\ 
		\hline 
	\end{tabular}
	\caption {Parámetros proporcionados por el usuario para los cubos de datos simulados}
\end{center}
\end{table}

Con el fin de representar las mediciones de manera cercana a la realidad, se escogió con la ayuda de los astrónomos un set de moléculas y sus respectivos isótopos. El criterio consistió en seleccionar moléculas con una estructura que no fuera en exceso complicada, de forma que no supusieran una complejidad innecesaria para el algoritmo. Además, este set de moléculas debía ser representativo de las moleculas que se esperan encontrar en los objetos astronómicos. Estas se muestran en la siguiente tabla.

\begin {table}[H]
\begin{center}
	\begin{tabular}{|c|c|c|}
		\hline Nombre & Fórmula &  Isótopos \\ 

		\hline 	Carbon Monoxide & 'CO' & 'COv=0','COv=1','13COv=0','C18O'\\
								&	   & 'C17O','13C17O','13C18O' \\
		
		\hline	Diazenylium & 'N2H' & 'N2H+v=0', 'N2D+', '15NNH+', 'N15NH+' \\
			
		\hline	Cyanide Radical & 'CN' & 'CNv=0', '13CN', 'C15N' \\
			
		\hline	Hydrogen Cyanide & 'HCN' & 'HCNv=0', 'HCNv2=1', 'HCNv2=2','HCNv3=1' \\
								 &       & 'HC15Nv=0', 'H13CNv2=1', 'H13CNv=0'\\ &       & 'HCNv1=1', 'HCNv3=1', 'DCNv=0'\\ &       & 'DCNv2=1', 'HCNv2=4',  'HCNv2=1 1-v2=4 0' \\
						
		\hline  Carbon Monosulfide & 'CS' & 'CSv=0', '13C34Sv=0', 'C36Sv=0'\\
						    	   &      & 'C34Sv=0', 'CSv=1-0', '13CSv=0'\\
							       &      & 'C33Sv=0', 'CSv=1', 'C34Sv=1' \\
			
		\hline	Thioxoethenylidene & 'CCS' & 'CCS', 'C13CS', '13CCS', 'CC34S' \\
			
		\hline	Hydrogen sulfide & 'H2S' & 'H2S', 'H2S', 'H234S', 'D2S' \\
			
		\hline	Thioformaldehyde & 'H2CS' & 'H2CS', 'H213CS', 'H2C34S' \\
			
		\hline	Sulfur Dioxide & 'SO2' & 'SO2v=0', '33SO2', '34SO2v=0', 'SO2v2=1' \\
			
		\hline	Sulfur Dioxide & 'OSO' & 'OS18O', 'OS17O' \\
			
		\hline	Formaldehyde & 'H2CO'  & 'H2CO', 'H2C18O', 'H213CO' \\
			
		\hline	Formylium & 'HCO'  & 'HCO+v=0', 'HC18O+', 'HC17O+', 'H13CO+' \\
			
			
		\hline	Cyanobutadiyne & 'HC5N' & 'HC5Nv=0', 'HC5Nv11=1', 'HCC13CCCN'\\
							   &        & 'HCCCC13CN', 'HCCC13CCN', 'H13CCCCCN'\\
							   &        & 'HC13CCCCN' \\
			
		\hline	Methanol & 'CH3OH' & 'CH3OHvt=0', '13CH3OHvt=0 ', 'CH318OH' \\
						 &      & 'CH3OHvt=1 ', '13CH3OHvt=1'\\

		\hline 
	\end{tabular}
	\caption {Conjunto de moléculas e isótropos con los que se realizaron las simulaciones.}
\end{center}
\end{table}
