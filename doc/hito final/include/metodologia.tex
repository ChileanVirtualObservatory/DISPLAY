\section{Metodología de Trabajo}

En la metodología de trabajo se detalla el proceso a través del cual el problema a solucionar fue concebido, los obstáculos que ha presentado y los pasos seguidos para hallar la solución al problema.

Este proyecto surge de la potencial capacidad de la minería de datos de ser utilizados como herramienta para la astronomía. La dificultad de los problemas astronómicos complejos utilizando minería de datos radica en que estos involucren una gran cantidad de datos a ser procesados.

ChiVO pondrá a disposición herramientas que utilizarán datos de ALMA, por lo que la cantidad de datos de mediciones astronómicas a procesar crecerá exponencialmente. Es por esto que una herramienta que automatice el procesamiento de dichos datos y entregue información útil y especializada a astrónomos será de gran utilidad.

La metodología se puede separar en varias etapas para resolver el problema en
cuestión:


\begin{description}
\item [Definición del problema] Identificación de problemas a resolver: Se debe acotar el problema, definir el tipo de datos que se utilizarán, en este caso la simulación de datos tipo ALMA. Esto último lleva a la siguiente etapa:

\item [Simulación de datos] Especificaciones: Se deben definir con los parámetros de la simulación, así como un subconjunto de moléculas, en conjunto con astrónomos. Para esto deben considerarse las limitaciones de la simulación.

\item [Estado del arte] Se estudiará el estado del arte en la clasificación de líneas espectrales. Se buscarán situaciones similares con el fin de aplicar técnicas adecuadas a este caso. Finalmente, para la codificación de espectros en variables que representen las moléculas presentes, se buscará una representación en base a trabajos previos en el área, como las aplicaciones para la determinación de compuestos existentes en espectros de Ranman \cite{howley_effect_2005}.

\item [Desarrollo de la solución] se realizará una iteración de soluciones al problema de codificación sparse, analizando el efecto de los parámetros en los modelos a utilizar y determinando la cantidad de parámetros necesarios para abarcar mayor complejidad en la identificación de los espectros.
\end{description}


