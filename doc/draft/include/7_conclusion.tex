\section{Conclusion} \label{sec:conclusions}

% Present Tense %
\begin{comment}
  - Remember hypothesis
  - Results support hypothesis
  - Reached objective proposed in the focus of the problem
  - With some exceptions, results archived
  - Extensibility
  - Next steps
\end{comment}

% Summary: Summarizing (briefly) from specific to general
The approach of the investigation is to identify emission lines through the reconstruction of an input signal.
The combination of representative basis vectors allows us to predict the presence of isotope energy state's lines along wavelength spectra. 

The hypothesis of this investigation can be summarized in as:
i) a set of basis vectors representing theoretical lines allows to reconstruct an input spectra, and
ii) the used combination of basis vectors gives useful information to identify the presence of emission lines along the spectra.

% Confirm the topic in the introduction
Results has shown support for the hypothesis and leaves room for improvements that increases its possibilities given the near coming of real data.
This data will allow to future investigators to train models and capture actively the patterns in data, its correlations and hidden latent variables.
Sparse coding technique allows to identify isotope energy states even with blending.
In addition, even when prediction fails, the mismatch correspond to an isotope of the predicted molecule.
With this, it is possible to give a basic notion of which is the molecular composition of the astronomical objects simulated.
For lines with higher miss match, astronomers can focus on analyze just complex cases and automatize detection of a majority of the spectral lines.

% Extensibility
A major issue in the algorithm elaboration is the lack of relative intensities patterns and co-presence dependence for isotope rotational sequences.
That makes the algorithm not to relay in co-presences and just try to find each isotope line independently.
Future extensibility from real data can be the inclusion of relationship between temperatures of lines belonging to the same isotope, or to learn the dependence of co-presence of lines, not for same isotopes, but for all molecules.
On that line, theoretical lines belonging to unknown molecules are an interesting case to cover.
The lack of direct common patterns between these lines could be learned and relationships discovered as a way to assign them to known isotopes. 

% Finally, how the topic relates to its context
% Next steps
The solution proposed resulted in a first approach to solve this problem.
Real data will give to researchers new tools to analyze and develop more complex models to make use of patterns that simulations do not allow us to use.
Certainly, future work can make use of a big amount of data available with the forward of ALMA project to apply more complex word representations and signal reconstruction models.