\section{Conclusion} \label{sec:conclusions}

% Present Tense %
\begin{comment}
  - Remember hypothesis
  - Results support hypothesis
  - Reached objective proposed in the focus of the problem
  - With some exceptions, results archived
  - Extensibility
  - Next steps
\end{comment}

% Summary: Summarizing (briefly) from specific to general
Our approach to identify emission lines is the reconstruction of an input signal.
The combination of representative basis vectors allows us to predict the presence of isotopes lines along the wavelength spectra.
The set of coefficients used to reconstruct the signal give us an idea of the presence of each isotope line.

This process can be summarized as two main hypothesis:
i) a set of basis vectors representing theoretical lines allows to reconstruct an input spectra, and
ii) the used combination of basis vectors gives useful information to identify the presence of emission lines along the spectra.

% Confirm the topic in the introduction
Results has shown support for the hypothesis, but leaves room for improvements that increases its possibilities given the near arrival of new real data.
This data will allow to future investigators to train models and capture actively the patterns in data, its correlations and hidden latent variables.

Sparse coding technique allows to identify isotope lines even when blending is present.
This gives a notion of the molecular composition of the astronomical object and allows astronomers to focus on complex cases.

% Extensibility
A major issue in the algorithm elaboration is the lack of information about relative intensities relationships and the co-presence dependence for lines of the same isotope.
That makes the algorithm to try to find each isotope line independently.
Future extensibility from real data can be:
i) the inclusion of relationship between temperatures of lines belonging to the same isotope, and 
ii) to learn the dependence of co-presence of lines, not just for the same isotopes, but for all molecules.
On that line, theoretical lines belonging to unknown molecules are an interesting case to cover.
The possible relationships between unidentified lines and known molecules could be used as a way to assign unknown lines to an isotope. 

% Finally, how the topic relates to its context
% Next steps
The solution proposed resulted in a first approach to solve this problem.
Real data will give to researchers new tools to analyze and develop more complex models to make use of patterns that simulations do not allow us to use.
Certainly, future work can make use of a big amount of data available with the forward of ALMA project to apply more complex word representations and signal reconstruction models.